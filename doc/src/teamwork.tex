\section{Teamwork}\label{Teamwork}
\subsection{Work distribution} \label{Work distribution}
\begin{center}
\begin{tabular}{|l|l |} 
    \hline 
    \textbf{xpomsa00:} & Team leader, symbol table, Top-Down parser, testing, documentation \\
    \hline
    \textbf{xcagal00:} & Top-Down parser, LL grammar, documentation\\
    \hline
    \textbf{xkolar77:} & Lexical analyzer, code generator, CI/CD setup, documentation \\
    \hline
    \textbf{xponec01:} & Precedence analysis, documentation \\
    \hline
\end{tabular}
\end{center}
The workload was evenly distributed, ensuring that each team member contributed equally, and each will receive the same amount of points (25 \% ).

\subsection{Communication \& Git}\label{Communication}
Communication among team members primarily occurred through \textit{Discord} or personal interactions.
During the course of the project, we employed the \textit{Git} version control system to manage the evolution of our codebase. GitLab served as our remote repository, providing a centralized platform for collaboration and version tracking. The integration of \textit{CI/CD} testing on \textit{GitLab} ensured that changes introduced to the codebase were systematically tested.

\subsection{Testing} \label{Testing}
Initially, unit tests were crafted to guarantee the correct behavior of individual modules. As the project progressed, integration tests and end-to-end tests were developed to ensure seamless integration and communication among these modules.

Additionally, trial submissions were employed, with achieving correctness rates of \textbf{40\%} and \textbf{65\%}, respectively. Towards the completion of the project, student's tests were incorporated for further validation. As mentioned earlier in \ref{Communication}, \textit{CI/CD} testing played a crucial role in verifying the correctness of individual pushed branches.